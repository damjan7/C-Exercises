\documentclass[a4paper]{article}
\usepackage{booktabs}
\usepackage[utf8]{inputenc}
\usepackage{amsmath}
\usepackage{amssymb}
\usepackage{wrapfig}
\usepackage{multirow}
\usepackage{graphicx}
\usepackage{listings}
\usepackage{ifthen}
\usepackage{verbatim}
\usepackage{array}
\usepackage{tikz}
\usetikzlibrary{decorations.pathreplacing}
\usepackage{lastpage}
\usepackage[normalem]{ulem}
\usepackage{fancyhdr}
\usepackage{graphicx}
\usepackage{amsmath}
\usepackage[export]{adjustbox}
\usepackage{tabularx}
\usepackage{subfigure}
\usepackage{algodat}
\usepackage[boxed,vlined,titlenotnumbered]{algorithm2e}
\usepackage{url}
\usepackage[all]{xy}
\usepackage{enumerate}
\usetikzlibrary{calc,shapes.multipart,chains,arrows}
\usepackage{pifont}


\definecolor{grL}{rgb}{0.827451,0.827451,0.827451}

%%%%%%%%%%%%%%%%%%%%%%%%%%%%%%%%%%%%%%%%%%%%%%%%%%%%%%%%%%%%%%%%%%%%%%%%%%%%%%%

\newcommand{\nil}{\texttt{NIL}}
\newcommand{\command}[1]{\texttt{#1}}
\newcommand{\pseudo}[2]{\rule{#1ex}{0ex}{#2}}

\renewcommand{\algorithmcfname}{Algo}
\setlength{\algomargin}{5pt}
\SetAlFnt{\small\texttt}
%\SetAlFnt{\small\ttfamily}
%\SetAlFnt{\small}
%\SetArgSty{texttt}
\SetKw{Break}{break}
\SetKwInput{KwNotation}{Notation}
\SetInd{0.2cm}{0.2cm}

%%%%%%%%%%%%%%%%%%%%%%%%%%%%%%%%%%%%%%%%%%%%%%%%%%%%%%%%%%%%%%%%%%%%%%%%%%%%%%%

\definecolor{mGreen}{rgb}{0,0.6,0}
\definecolor{mGray}{rgb}{0.5,0.5,0.5}
\definecolor{mPurple}{rgb}{0.58,0,0.82}
\definecolor{backgroundColour}{rgb}{0.95,0.95,0.92}

\lstdefinestyle{CStyle}{
    backgroundcolor=\color{backgroundColour},   
    commentstyle=\color{mGreen},
    keywordstyle=\color{magenta},
    numberstyle=\tiny\color{mGray},
    stringstyle=\color{mPurple},
    basicstyle=\footnotesize,
    breakatwhitespace=false,         
    breaklines=true,                 
    captionpos=b,                    
    keepspaces=true,                 
    numbers=left,                    
    numbersep=5pt,                  
    showspaces=false,                
    showstringspaces=false,
    showtabs=false,                  
    tabsize=2,
    language=C
}

%%%%%%%%%%%%%%%%%%%%%%%%%%%%%%%%%%%%%%%%%%%%%%%%%%%%%%%%%%%%%%%%%%%%%%%%%%%%%%%

\def\checkmark{\tikz\fill[scale=0.4](0,.35) -- (.25,0) -- (1,.7) -- (.25,.15) -- cycle;} 

%%%%%%%%%%%%%%%%%%%%%%%%%%%%%%%%%%%%%%%%%%%%%%%%%%%%%%%%%%%%%%%%%%%%%%%%%%%%%%%

\newcommand{\cmark}{\ding{51}}%
\newcommand{\xmark}{\ding{55}}%

%%%%%%%%%%%%%%%%%%%%%%%%%%%%%%%%%%%%%%%%%%%%%%%%%%%%%%%%%%%%%%%%%%%%%%%%%%%%%%%

\newcommand{\showsolutions}{true}
\DeclarePairedDelimiter\ceil{\lceil}{\rceil}
\DeclarePairedDelimiter\floor{\lfloor}{\rfloor}

\newcommand{\solution}[1]{
  \ifthenelse{\equal{\showsolutions}{true}}
             {{\color{blue}{#1}}}
             {\newpage}}

\definecolor{grL}{rgb}{0.827451,0.827451,0.827451}

\newcounter{taskNo}
\setcounter{taskNo}{0}
\newcommand{\showTaskNo}{\arabic{taskNo}}
\newcommand{\incTaskNo}{\addtocounter{taskNo}{1}}

%%%%%%%%%%%%%%%%%%%%%%%%%%%%%%%%%%%%%%%%%%%%%%%%%%%%%%%%%%%%%%%%%%%%%%%%%%%%%%%

\begin{document}

% Topics: Hash Tables & Hash Funcions
\solutiontitle{10}{May 06, 2019}

\section*{Hash Tables, Hash Functions\hfill}

\paragraph{Task 1.} 
The hash table $T[0\dots 10]$ stores integers and uses chaining to
resolve conflicts. Illustrate the hash tables after the values 5,
19, 27, 15, 30, 34, 26, 12, and 21 have been inserted (in that order)
using the following hash functions:

\begin{enumerate}
  \item $h_1(k) = (k + 7)\mbox{ mod }11$ \quad \textbf{(Division Method)}
  
  \solution{

\begin{figure}[!h]
  \centering
  \begin{minipage}[t]{0.2\textwidth}
    \centering
    \underline{\color{blue}{Hashtable ($h_1$)}}

    \begin{tikzpicture}[grow=east,arrows=-,level distance=20mm]
      \tikzstyle{every node}=[draw]
      \node at (0mm,50mm) {0} child {node[rounded corners]{26}
        child {node[rounded corners]{15}}};
      \node at (0mm,45mm) {1} child {node[rounded corners]{27}
      child {node[rounded corners]{5}}};
      \node at (0mm,40mm) {2}; 
      \node at (0mm,35mm) {3}; 
      \node at (0mm,30mm) {4} child {node[rounded corners]{30}
      child {node[rounded corners]{19}}}; 
      \node at (0mm,25mm) {5};
      \node at (0mm,20mm) {6} child {node[rounded corners]{21}};
      \node at (0mm,15mm) {7};
      \node at (0mm,10mm) {8} child {node[rounded corners]{12}
      child {node[rounded corners]{34}}};
      \node at (0mm, 5mm) {9};
      \node at (0mm, 0mm) {10};
    \end{tikzpicture}
  \end{minipage}
\end{figure}
  }
  \item $h_2(k) = \left \lfloor{8 (k \cdot 0.618\mbox{ mod }1)}\right \rfloor$ \quad \textbf{(Multiplication Method)}


\solution{
\begin{figure}[!h]
  \centering
  \begin{minipage}[t]{0.2\textwidth}
    \centering
    \underline{\color{blue}{Hashtable ($h_2$)}}

    \begin{tikzpicture}[grow=east,arrows=-,level distance=20mm]
      \tikzstyle{every node}=[draw]
      \node at (0mm,50mm) {0} child {node[rounded corners]{26}
      child {node[rounded corners]{34}
      child {node[rounded corners]{5}}}};
      \node at (0mm,45mm) {1};
      \node at (0mm,40mm) {2} child {node[rounded corners]{15}};
      \node at (0mm,35mm) {3} child {node[rounded corners]{12}};
      \node at (0mm,30mm) {4} child {node[rounded corners]{30}};
      \node at (0mm,25mm) {5} child {node[rounded corners]{27}
      child {node[rounded corners]{19}}};
      \node at (0mm,20mm) {6};
      \node at (0mm,15mm) {7} child {node[rounded corners]{21}};
      \node at (0mm,10mm) {8};
      \node at (0mm, 5mm) {9};
      \node at (0mm, 0mm) {10};
    \end{tikzpicture}
  \end{minipage}
\end{figure}

}
\end{enumerate}
\newpage

\paragraph{Task 2.}
The hash table T[0...10] may contain up to 11 elements and uses open
addressing.  Draw an image of the hash table after the keys 5, 19, 27,
15, 30, 34, 26, 12, and 21 have been inserted (in that order) using
the following hash functions:
\noindent For each table indicate the longest probe sequence with the
corresponding key.
\begin{enumerate}
  \item $h_1(k, i) = (k + i)\mbox{ mod }11$
  
  \solution{
  \textbf{Linear probing:} largest probe is 4 for key(26)\\

  \begin{center}
    \begin{tabular}{|m{1cm}||m{1cm}|@{}m{0cm}@{}}
      \hline
      \centering  \textbf{Entry} & \centering  \textbf{Value} & 
      \\ [0.3cm]
      \hline  
      \centering 0 & \centering  & 
      \\ [0.3cm]
      \hline
      \centering \solution{1} & \centering \solution{34} & 
      \\ [0.3cm]
      \hline
      \centering \solution{2} & \centering  \solution{12}&
      \\ [0.3cm]
      \hline
      \centering \solution{3} & \centering & 
      \\ [0.3cm]
      \hline
       \centering \solution{4} & \centering \solution{15} & 
      \\ [0.3cm]
      \hline
       \centering \solution{5} & \centering \solution{5}& 
      \\ [0.3cm]
      \hline
       \centering \solution{6} & \centering \solution{27} & 
      \\ [0.3cm]
      \hline
      \centering \solution{7} & \centering \solution{26} & 
      \\ [0.3cm]
      \hline
      \centering \solution{8} & \centering \solution{19} & 
      \\ [0.3cm]
      \hline
      \centering \solution{9} & \centering \solution{30} & 
      \\ [0.3cm]
      \hline
      \centering \solution{10} & \centering \solution{21} & 
      \\ [0.3cm]
      \hline
    \end{tabular} 
  \end{center}}
  \item $h_2(k, i) = (k\mbox{ mod }11 + i (k\mbox{ mod }7))\mbox{ mod }11$


\solution{
\textbf{Double hashing probing:} largest successful probe is 1 for insert of keys (27, 30, 26, 12)\\

\begin{center}
  \begin{tabular}{|m{1cm}||m{1cm}|@{}m{0cm}@{}}
    \hline
    % 34 & 12 & & 15 & 5 & 27 & 26 & 19 & 30 & 21
    \centering  \textbf{Entry} & \centering  \textbf{Value} & 
    \\ [0.3cm]
    \hline  
    \centering 0 & \centering  \solution{27}& 
    \\ [0.3cm]
    \hline
    \centering \solution{1} & \centering \solution{34} & 
    \\ [0.3cm]
    \hline
    \centering \solution{2} &  &
    \\ [0.3cm]
    \hline
    \centering \solution{3} & & 
    \\ [0.3cm]
    \hline
     \centering \solution{4} & \centering \solution{15} & 
    \\ [0.3cm]
    \hline
     \centering \solution{5} & \centering \solution{5}& 
    \\ [0.3cm]
    \hline
     \centering \solution{6} & \centering \solution{12} & 
    \\ [0.3cm]
    \hline
    \centering \solution{7} & \centering  & 
    \\ [0.3cm]
    \hline
    \centering \solution{8} & \centering \solution{19} & 
    \\ [0.3cm]
    \hline
    \centering \solution{9} & \centering \solution{26}& 
    \\ [0.3cm]
    \hline
    \centering \solution{10} & \centering \solution{30} & 
    \\ [0.3cm]
    \hline
  \end{tabular} 
\end{center}

}

\end{enumerate}



\paragraph{Task 3.}
Write a C-program where a hash table consisting of $m$ slots of positive integers is implemented.
\textbf{Double hashing} should be used to resolve conflicts. The hash-function to be used
is:
\begin{align*}
	h(k,i) &= (h_1(k) + i* h_2(k))\ mod\ m \\
	h_1(k) &=  (k\ mod\ m) + 1 \\
	h_2(k) &=   m^{\prime} - (k\ mod\ m^{\prime}) \\
	m^{\prime} &= m-1 
\end{align*}



Your program should include the following:

\begin{itemize}
	\item The constant $m$ corresponding to
      the size of the hash table.
  
  \solution{\lstinputlisting[language=c, firstline=4, lastline=4]{code/task3.c}}

	
	\item The function \textit{void init(int A[])} that fills
	      all slots of the hash table $A$ with the value 0.
  
  \solution{\lstinputlisting[language=c, firstline=6, lastline=11]{code/task3.c}}      
        
	\item The hashing function \textit{int h(int k, int i)} that
      receives the key $k$ and the probe number $i$ and 
      returns the hashed key. 
      
  \solution{\lstinputlisting[language=c, firstline=13, lastline=18]{code/task3.c}}

	\item The function \textit{void insert(int A[], int key)} that
      inserts $key$ into the hash table $A$.
    
  \solution{\lstinputlisting[language=c, firstline=33, lastline=40]{code/task3.c}}

	\item The function \textit{int search(int A[], int key)} that
		  returns -1 if the requested $key$ was not found in the
		  hash table $A$. Otherwise, it should return the index
      of the requested hash key in the table.
    
  \solution{\lstinputlisting[language=c, firstline=20, lastline=31]{code/task3.c}}

	\item The function \textit{void printHash(int A[])} that prints the table
    size and all non-empty slots of the hash table $A$ accompanied with their index and the key.
    
  \solution{\lstinputlisting[language=c, firstline=42, lastline=50]{code/task3.c}}

\begin{center}
\begin{minipage}[t]{0.2\textwidth}
\centering
\underline{\textbf{Hashtable}}

\begin{tikzpicture}[grow=east,arrows=-,level distance=20mm]
\tikzstyle{every node}=[draw]
\node at (0mm,50mm) {0} child {node[rounded corners]{1315}};
\node at (0mm,45mm) {1} child {node[rounded corners]{2002}};
\node at (0mm,40mm) {2} child {node[rounded corners]{2001}};
\node at (0mm,35mm) {3} child {node[rounded corners]{2000}};
\node at (0mm,30mm) {4};
\node at (0mm,25mm) {5} child {node[rounded corners]{1313}};
\node at (0mm,20mm) {6} child {node[rounded corners]{1314}};
\end{tikzpicture}
\end{minipage}
\hspace{1.5cm}
\begin{minipage}[t]{0.5\textwidth}
\underline{\textbf{Output printHash}}		
\begin{verbatim}
Table size: 7
i: 0    key: 1315
i: 1    key: 2002
i: 2    key: 2001
i: 3    key: 2000
i: 5    key: 1313
i: 6    key: 1314
\end{verbatim}
\end{minipage}
\end{center}

\end{itemize}

Mathematical functions available in the library \lstinline!math.h! can be used
and don't need to be redefined. For testing purposes, use a table size
of 7, add the values 1313, 1314, 1315, 2000, 2001 and 2002 and print your
hashtable. Search for the values 2000, 10, 1314, 1313 and 337 and print the results.

\paragraph{Task 4.}

Write a C-program where a hash table consisting of $m$ slots of positive integers is implemented.
\textbf{Chaining} should be used for collision resolution.
 The hash function to be used is
$h(k) = \lfloor m (kA \mbox{ mod }1) \rfloor$ where
$A=(\sqrt{7}-1)/2$ and $kA \mbox{ mod } 1$ returns the
fractional part of $kA$. i.e. kA - $\lfloor kA \rfloor$. Each slot is implemented as a linked
list to resolve conflicts. A
node of a hash table slot is defined as follows:

\begin{center}
  \begin{minipage}[t]{0.425\textwidth}
    \begin{lstlisting}[language=C]
struct element {
  int val;
  struct element *next;
};
\end{lstlisting}
  \end{minipage}
\end{center}

The global hash table \texttt{H} is defined, as follows:

\begin{center}
  \begin{minipage} [t]{0.425\textwidth}
    \begin{lstlisting}[language=C] 
struct element* H[m];
\end{lstlisting}
  \end{minipage}
\end{center}

Implement the following functions:

\begin{itemize}
\item \textit{void init()}, which initializes hash table $H$.

\solution{\lstinputlisting[language=c, firstline=14, lastline=18]{code/task4.c}}

\item \textit{int h(int val)}, which gets a value $val$ as input and
  returns the hashed key.

\solution{\lstinputlisting[language=c, firstline=20, lastline=23]{code/task4.c}}

\item \textit{struct element* search(int val)}, which returns NULL if
  the requested $val$ is not found in hash table H.  Otherwise, it
  returns a pointer to the node with the requested $val$.

\solution{\lstinputlisting[language=c, firstline=34, lastline=43]{code/task4.c}}

\item \textit{void insert(int val)}, which inserts value $val$ into
  hash table $H$, if it does not already exist in the table (no
  duplicates should exist).

\solution{\lstinputlisting[language=c, firstline=25, lastline=32]{code/task4.c}}

\item \textit{void printHash()}, which prints the values of all
  non-empty slots of the hash table $H$.

\solution{\lstinputlisting[language=c, firstline=45, lastline=61]{code/task4.c}}
\end{itemize}

\begin{center}
\begin{minipage}[t]{0.2\textwidth}
\centering
\underline{\textbf{Hashtable}}

\begin{tikzpicture}[grow=east,arrows=-,level distance=20mm]
\tikzstyle{every node}=[draw]
\node at (0mm,50mm) {0};
\node at (0mm,45mm) {1};
\node at (0mm,40mm) {2} child {node[rounded corners]{1342} child {node[rounded corners]{111}}};
\node at (0mm,35mm) {3} child{node [rounded corners]{21231}};
\node at (0mm,30mm) {4}; 
\node at (0mm,25mm) {5};
\node at (0mm,20mm) {6} child {node[rounded corners]{63} child {node[rounded corners]{5568}
child {node[rounded corners]{1113}}}};
\node at (0mm,15mm) {7} child {node[rounded corners]{10112}};
\end{tikzpicture}
\end{minipage}
%
\hspace{2.5cm}
\begin{minipage}[t]{0.5\textwidth}
\underline{\textbf{Output printHash}}		
\begin{verbatim}
Table size: 8
i: 2     val: -> 1342 -> 111
i: 3     val: -> 21231
i: 6     val: -> 63 -> 5568 -> 1113
i: 7     val: -> 10112
\end{verbatim}
\end{minipage}
\end{center}

Mathematical functions that are available via library
\lstinline!math.h! can be used and don't need to be redefined.  For
testing purposes, use a hash table of size 8, add the values 111,
10112, 1113, 5568, 63, 1342 and 21231, and print your hash table.
Search for values 1, 10112, 1113, 5568, 337 and print the results.

\end{document}


